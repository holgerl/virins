VR-landsbyen har v�rt veldig resultatorientert, og oppgavene har kun v�rt innenfor datateknologi. Gruppa v�r har fungert meget bra n�r det gjelder samarbeidsprosessen, men vi har hatt en del problemer med oppgavel�sningen. Problemene kom av gruppas manglende datakompetanse. Dette har naturlig nok f�rt til en del frustrasjon, men vi er glade for at dette ikke har f�rt til noen interne konflikter i gruppen. Vi har ubevisst rettet v�r frustrasjon utad istedenfor innad i gruppen. 

Vurderingen v�r av gruppen er at vi p� de fleste omr�der har fungert som et godt team. Ett ankepunkt er kanskje mangel p� beslutningsdyktighet. Vi f�ler ikke at gruppa er satt sammen av den riktige faglige kompetansen for � kunne l�se de oppgavene som ble lagt fram i VR- landsbyen optimalt. �Men samtidig mener vi man som profesjonell skal kunne fungere i et team som best�r av medlemmer med tverrfaglig bakgrunn. Gjennom dette prosjektet har vi sett at vi likevel klarte � f� til ganske mye, og at vi tilegnet oss mye ny l�rdom. Dette gj�r at vi neste gang ikke vil se like m�rkt p� en vanskelig oppgave hvor vi mangler viktige forkunnskaper. 
� 
N�r det gjelder samholdet og trivselen i gruppa, har den hele veien v�rt veldig god. Samholdet har holdt seg like godt b�de i medgang og motgang, og dette har ogs� syntes godt utad. Dette har blant annet v�rt takket v�re sosial sammenkomst og teambuilding fra fasilitatorene og landsbyleders side. Dette f�rte til �kt selvtillit og viljen til � st� p� ble merkbart bedre etter dette. 

Vi har i Eksperter i Team ikke f�tt noen erfaringer i � jobbe med personer som er vanskelig � samarbeide med. Vi har heller ikke opplevd andre store konflikter under gruppearbeidet. Vi har derimot h�stet erfaring i hvordan en gruppe p� best mulig m�te kan utnytte de resursene som hvert enkelt gruppemedlem utgj�r og kunnskapen de besitter. Vi har sett at det er viktig � kartlegge de enkeltes sterke sider p� et tidlig tidspunkt slik at disse kan komme gruppen raskt til gode. Vi har gjennomf�rt adferdstest og satt oss inn i en del gruppeteori som vi ikke har reflektert noe s�rlig over ved tidligere teamarbeid. Vi synes alle at vi har f�tt bedre innblikk i hvordan vi selv opptrer i et team. Vi tror erfaringene v�re fra EiT vil gj�re oss mer bevisste p� teamprosess ved senere prosjekter og i arbeidslivet. 

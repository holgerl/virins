\section{Design av l\o  sningen}

	\subsection{Overordnet design (Holger)}
	
		\begin{figure}[h]
		\centering
		\includegraphics[width=\textwidth]{graphics/main_design.png}
		\caption{Overordnet design}
		\label{fig:main_design}
		\end{figure}
		
		Den overordnede designen er en pipeline som begynner med lys fra LEDs og ender opp med visning av en 3D-scene. Figur \ref{fig:main_design} illustrerer dette. Vi ser at alt begynner med infrar\o  de LEDs som er festet p\aa \space hodet til brukeren. Disse sender lys som filmes av kameraet. Bildene fra kameraet leses av kode som finner punktene p\aa \space bildet som senteret til hvert filmet LED. Disse koordinatene sendes til kode som bruker dem til \aa \space \space beregninge posisjonen til hodet. Denne posisjonen sendes til kode som viser en 3D-scene p\aa \space skjermen, hvor synsvinkelen passer posisjonen til hodet.
	
	\subsection{Design av LED-headset (Jon)}
	
		
	
	\subsection{Design av punktfinneren (Holger)}
	
		
	
	\subsection{Design av posisjonsfinneren og 3D-viseren (Vegar)}
	
		
	
\section{Implementasjon av l\o  sningen}

	\subsection{Valg av kamera}
	
		Kriterier vi ser etter:
		
		\begin{itemize}
			\item Bildebrikke som kan fange opp infrar\o dt lys
			\item Stor n\o yaktighet p\aa avstand
			\item St\o tte for live feed til en datamaskin
			\item Manuell fokus
			\item Pris
		\end{itemize}
		
		 
		For \aa \space f\aa \space en fornuftig pris p\aa \space kameraet unders\o kte vi vanlige full hd (high definition) kamera. Vi s\aa \space etter et kamera som fulgte standarden 1080P. Dette er en oppl\o sning p\aa 1920 linjer horisontalt, 1080 linjer vertikalt, og gir h\o y n\o yaktighet p\aa avstand. P st\aa r for progressive. Det betyr at kameraet tar et fullt bilde i motsetning til interlaced, hvor det bare tar annenhver linje. Vi \o nsker progressive for \aa \space f\aa mindre bildebehandling enn ved interlaced. For \aa \space f\aa live feed fra kameraet er vi avhengig av \aa \space overf\o re store datamenger fra kameraet til datamaskinen. Analog overf\o ring er ikke aktuelt p\aa \space grunn av kvalitetstapet i bildet. Hver kanal gir ut 1byte per piksel, og det er 3 kanaler for r\o d, gr\o nn og bl\aa . Kameraet gir ut 25 bilder i sekundet, som gir 150 USB 2.0 kan overf\o re opp til 480 megabit, og dette er for lite til \aa \space \o verf\o re en ukomprimert live feed i full oppl\o sning. Derfor er vi avhengige av HDMI output p\aa kameraet. En stor del av kameraene tar i bruk en 3CCD bildesensor. CCD brikkene fanger opp en del av det infrar\o de spekteret i tillegg til synlig lys, derfor settes det p\aa et infrar\o dt filter foran sensoren. Vi tar utgangspunkt i at vi fjerner dette filteret. Dette vil f\o re til at autofokusen ikke vil fungere lenger, og vi er derfor avhengig av manuell fokus. CMOS bildebrikker er et annet alternativ da disse ogs\aa fanger opp infrar\o dt lys.
		
		Valget av kamera falt til slutt p\aa et Panasonic HDC-SD9 kamera. Dette har etter spesifikasjonene 3CCD bildesensor, HDMI utgang, manuell fokus, 1080P bilde og lav pris. Etter noe fram og tilbake fikk vi servicemanualen til kameraet fra Panasonic. Den beskriver trinnvis hvordan IR-filteret kan fjernes. Vi fikk fjernet IR-filteret, men det viste seg at manuell fokus ikke fungerte som forventet. Det var en automatisert funksjon som sto for fokuset, slik at dette ikke kunne stilles inn riktig uten filteret. Optimalt burde vi erstattet filteret med en glassbit, men vi valgte \aa \space erstatte filteret med gjennomsiktig pvc plast, slik at lysbrytningen skulle ligne p\aa \space den gjennom IR-filteret. De optiske egenskapene til plasten er ikke spesielt bra, men vi ender likevel opp med et mye bedre bilde p\aa grunn av at bildet ikke lenger er like langt ute av fokus. For v\aa r applikasjon er dette bildet mer enn nok til \aa \space tracke IR lys i rommet. For \aa \space lettere kunne tracke lyset fra dioden har vi tatt i bruk et filter som blokkerer alt lys utenom infrar\o dt. REF: (OPTIR 1.0 NG 305 X 100 fra Instrument Plastics. Kj\o pt via http://www.farnell.com (varenr 177143))
		
	\subsection{Valg av IR-sender}
		
		Kriterier vi ser etter:
	
		\begin{itemize}
			\item Lett \aa \space bruke
			\item Liten vekt
			\item Minst mulig i veien
			\item H\o y intensitet p\aa lyset
		\end{itemize}

		M\aa let var \aa \space finne en l\o sning som ikke er i veien for brukeren. Optimalt sett s\aa trenger ikke brukeren ta p\aa \space seg noe ekstra utstyr. For \aa \space kunne tracke brukeren trengs det minimum to lyskilder og 1 kamera, eller to kamera og en lyskilde. Vi bestemte oss for \aa \space bruke et tr\aa dl\o st headset, da dette ikke er spesielt tungvindt \aa \space ha p\aa. Dette har ogs\aa et innebygd batteri som vi kan koble lyskilder p\aa . For \aa \space kunne skille rotasjon fra avstand trengs det en tredje diode. Denne kan festes p\aa \space mikrofonen.

		Vi gikk til innkj\o p av et Logitech Clearchat tr\aa dl\o st headset, og 3 IR dioder [SFH 415] med h\o y intensitet. IR diodene var spesifisert til \aa \space t\aa le opp mot 1 Ampere. Batteriet p\aa headsettet gir ut 3,7V. M\aa let var \aa \space f\aa \space mest mulig intensitet ut, slik at lyset er synlig lengst mulig bort, og slik at dioden ikke trenger peke direkte p\aa kameraet for \aa \space kunne registreres. Diodene vil fungere som en kortslutning om de festes rett p\aa batteriet, s\aa for \aa \space begrense str\o mmen gjennom dem trengs det motstand i serie. Vi dimensjonerte oppsettet, med en motstand p\aa 47? foran dioden, slik at det g\aa r under 100mA gjennom hver diode.

		Det viste seg at kameraet klarte \aa \space detektere diodene bedre enn forventet, som gj\o r at diodestr\o mmen er kraftig overdimensjonert, noe som g\aa r hardt utover batterikapasiteten til headsettet.
Dette er fagrapporten til Team 4 i Eksperter i Team, VR-Landsbyen. Den beskriver det faglige produktet som ble utviklet i l�pet av semesteret. Produktet er en l�sning for virtuell virkelighet, og denne rapporten inneholder resultatene av et forstudie, kravspesifikasjonen til l�sningen sammen med beskrivelse design og implementasjon og til slutt en presentasjon av resultatet av prosjektet.

\section{Problemstilling}

	Oppgaven som gruppa valgte var "Inexpensive Tracked VR". Dette var fordi gruppa synes denne oppgaven hadde h�yeste "wow-faktor", og fordi gruppa f�lte at kompentansen for � l�se den var tilstede. Problemstillingen vi kom frem til lyder som f�lger:
	
	\setlength{\parskip}{-0.15in}
	\begin{center}
		\em Er det mulig � lage en rimelig og brukbar l�sning \\
		for tracking av hode til bruk i virtuell virkelighet?
	\end{center}
	
	Tracking av hode betyr at man beregner posisjonen hodet til brukeren har i rommet. Med rimelig menes at systemets totale kostnad skal v�re under 10  000 NOK, og med brukbar menes det at systemet er lett � sette opp og bruke, selv for personer med lite erfaring med data.
	
	\setlength{\parskip}{0.1in}
	
	Gruppa ble enige om � bruke et headset til � feste infrar�de dioder p�, spore dem ved hjelp av et kamera, og ut fra dette beregne posisjonen til hodet som bruker headsettet.

\section{Ressurser}

	Dette er krevende prosjekt som trenger personer med gode faglige og teknologiske kunnskaper innen data, elektronikk og petroleum. Og med personer i gruppa innefor alle disse faglige feltene ble prosjektet gjennomf�rbart. Vi fikk ogs� mye hjelp fra ressurssterke personer utenfor gruppa innen data og elektronikk og vi fikk god hjelp fra Omega- og IPT-verkstedet.
	
	Dette prosjektet er ganske omfattende og kostnadskrevende med innkj�p av dyrt elektronisk utstyr. Derfor er samarbeidet og den �konomiske st�tten fra StatoilHydro sv�rt viktig. Det er blitt gjort en del innkj�p av utstyr i forbindelse med prosjektet selv om ikke alt er brukt i v�r endelige prototyp. Institutt for Petroleumsteknologi har gitt oss tilgjenglighet til mye avansert datautstyr som har v�rt viktig for gjennomf�relsen av prosjektet.
	
\section{Teamet}

	Vi best�r av 5 studenter p� NTNU. To g�r siv.ing. datateknikk, en g�r siv.ing. elektronikk, en g�r siv.ing. petroleumsteknologi og en g�r medisin. Vi er fire stykker fra 4. klasse og en fra 5., og alderen er fra 22 til 24 �r.

	\begin{figure}[h]
	\centering
	\includegraphics[width=0.60\textwidth]{graphics/people.jpg}
	\caption{Team 4. (Fra venstre) Jon Skarpetveit, Vegar Neshaug, Rahele Zarabi, Holger Ludvigsen, Kristoffer Selboe}
	\label{fig:people}
	\end{figure}